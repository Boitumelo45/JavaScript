\documentclass[10pt, a4paper, twocolumn]{article}

\input{structure.tex}
\title{JavaScript}

\author{
	\authorstyle{Boitumelo Phetla\textsuperscript{1}} % Authors
	\newline\newline % Space before institutions
	\textsuperscript{1}\institution{PluralSight Google ScholarShip}
}


\date{\today} % Add a date here if you would like one to appear underneath the title block, use \today for the current date, leave empty for no date


\begin{document}
\maketitle % Print the title
\thispagestyle{firstpage} % Apply the page style for the first page (no headers and footers)

%----------------------------------------------------------------------------------------
%	ABSTRACT
%----------------------------------------------------------------------------------------

\lettrineabstract{JavaScript,  is a lightweight interpreted or just-in-time compiled programming language with first-class functions. While it is most well-known as the scripting language for Web pages, many non-browser environments also use it, such as Node.js, Apache CouchDB and Adobe Acrobat.}

%----------------------------------------------------------------------------------------
%	ARTICLE CONTENTS
%----------------------------------------------------------------------------------------

\section{Simplistic JavaScript 1}

\subsection{Command-line based programming}

A simple project:

\begin{lstlisting}

$bash: touch {index.html,script.js,style.css}
$bash: tree
	__________	index.html
	__________	script.js
	__________	style.css


\end{lstlisting}

Include the script (\textbf{javascript}) and the page styling script (\textbf{cascading stylesheet}) files into the \textit{index.html}.

\begin{lstlisting}
<!DOCTYPE>
	<html>
		<head>
				<script src="path/*.js"></script>
				<link rel="stylesheet" href="path/*.css">
		</head>
				<body>
						<div>
								<header></header>
						</div>
							<div><!-- body --></div>
						<div>
							<footer></footer>
						</div>
				</body>
	</html>

\end{lstlisting}

Add some simple HTML markup code and launch a live-server of the code.

\begin{lstlisting}
	<!DOCTYPE>
	<html>
			<head>
					<script src="script.js"></script>
					<link rel="stylesheet" href="style.css">
			</head>
					<body>
							<div id="header">
									<h1>Welcome to JavaScript</h1>
							</div>
					</body>
	</html>
\end{lstlisting}

Launch the command-line (Terminal)

\begin{lstlisting}
$bash: live-server
\end{lstlisting}

\begin{figure}[h!]
	\includegraphics[width=\linewidth]{liveserver.png} % Figure image
	\caption{Live-server} % Figure caption
	\label{ls} % Label for referencing with \ref{bear}
\end{figure}

\newpage
\subsection{\href{http://plnkr.co}{Plunker}}

Or create an account on  \href{http://plnkr.co/edit/?p=catalogue}{Plunker}.  Plunker sets up your working environment for you.

\begin{figure}[h!]
	\includegraphics[width=\linewidth]{plunker.png} % Figure image
	\caption{Plunker} % Figure caption
	\label{plnkr} % Label for referencing with \ref{bear}
\end{figure}

\subsection{\href{https://electronjs.org}{Electron}}

Watch this video \href{https://www.youtube.com/watch?v=8YP_nOCO-4Q&feature=youtu.be}{Electron}.

\begin{lstlisting}
# Clone the Quick Start repository
$ git clone https://github.com/electron/electron-quick-start

# Go into the repository
$ cd electron-quick-start

# Install the dependencies and run
$ npm install && npm start
\end{lstlisting}

\begin{figure}[h!]
	\includegraphics[width=\linewidth]{electron.png} % Figure image
	\caption{Electron} % Figure caption
	\label{elec} % Label for referencing with \ref{bear}
\end{figure}

\begin{lstlisting}
$bash: mkdir Electron1; cd Electron1; npm init
  1 {
  2   "name": "electron1",
  3   "version": "1.0.0",
  4   "description": "First App",
  5   "main": "index.js",
  6   "scripts": {
  7     "test": "echo \"Error: no test specified\" && exit 1"
  8   },
  9   "keywords": [
 10     "Electron"
 11   ],
 12   "author": "Boitumelo Phetla",
 13   "license": "ISC"
 14 }
\end{lstlisting}

At this point, you'll need to install electron itself. The recommended way of doing so is to install it as a development dependency in your app, which allows you to work on multiple apps with different Electron versions. To do so, run the following command from your app's directory:

\begin{lstlisting}
$bash: npm install --save-dev electron
$bash: tree -L 1
			.
			|____________node_modules
			|____________package-lock.json
			|____________package.json

1 directory, 2 files
\end{lstlisting}

All APIs and features found in Electron are accessible through the electron module, which can be required like any other Node.js module:

\begin{lstlisting}
const electron = require('electron')
\end{lstlisting}

To avoid any huddles, try this simple example.

\begin{lstlisting}
# Clone the repository
$ git clone https://github.com/electron/electron-quick-start
# Go into the repository
$ cd electron-quick-start
# Install dependencies
$ npm install
# Run the app
$ npm start
\end{lstlisting}

\subsection{\href{https://www.meteor.com/install}{Meteor}}


\begin{figure}[h!]
	\includegraphics[width=\linewidth]{meteor.png} % Figure image
	\caption{Meteor} % Figure caption
	\label{mtr} % Label for referencing with \ref{bear}
\end{figure}

To create the app, open your terminal and type:

\begin{lstlisting}
$bash: meteor create simple-todos

output:

Created a new Meteor app in 'simple-todos'.

To run your new app:
  cd simple-todos
  meteor

If you are new to Meteor, try some of the learning resources here:
  https://www.meteor.com/tutorials

To start with a different app template, try one of the following:

  meteor create --bare    # to create an empty app
  meteor create --minimal # to create an app with as few Meteor packages as possible
  meteor create --full    # to create a more complete scaffolded app

\end{lstlisting}


%------------------------------------------------

\subsection{Coding in JavaScript}

\subsubsection{Variables}

\begin{lstlisting}
"use strict";
//let is accessible in the code block where it is used
let firstName = "John Doe"; //camelCasing
console.log(firstName);

/*Output*/
$bash: node let.js
John Doe
\end{lstlisting}

\subsubsection{Global variable, function, Operators}
\begin{lstlisting}
"use strict";

//A = P(1 + rt)
let r = 10.5, t = 5, p = 200;

var A = (r,t,p) => {
  return p*(1 + (r/100)*t);
}

let interest = A(r,t,p);
console.log("R200 (interest in 5 years at at interest rate of 10.5% = R" + interest + "-00)");

\end{lstlisting}

\subsubsection{Simple function}

\begin{lstlisting}
"use strict";

//A = P(1 + rt)
let r = 10.5, t = 5, p = 200;

//function definition (without using arrow function)
var A = function(r,t,p){
    return p*(1 + (r/100)*t);
}

console.log(A(r,t,p));

\end{lstlisting}

\subsubsection{Variables and block code}

\begin{lstlisting}
"use strict";

array = [1,2,3,4,5];
var count = 0;
let counter = 0;

for(let i = 0; i < array.length; i++){
    count += array[i];
    counter += i;
    if(count > 5){
        var num1 = count*5;     //accessible outside this code block
        let num2 = num1*5;      //only accessible within this code block
        console.log("num1: ", num1, ',', 'num2: ', num2);
    }
    //console.log("xnum1: ", num1, ',', 'xnum2: ', num2);
    try{
        console.log("xnum1: ", num1, ',', 'xnum2: ', num2);
    }catch{
        console.log('xnum1: ', num1);   //<-- accessing num1
        console.log('xnum2: ', 'This will not print because it is not accessible'); //<-- can't access num2
    }

}
console.log('count: ', count, ',' , 'counter: ',counter); //<-- both count and counter are accessible because they are in the same code block

/*Output*/
xnum1:  undefined
xnum2:  This will not print because it is not accessible
xnum1:  undefined
xnum2:  This will not print because it is not accessible
num1:  30 , num2:  150
xnum1:  30
xnum2:  This will not print because it is not accessible
num1:  50 , num2:  250
xnum1:  50
xnum2:  This will not print because it is not accessible
num1:  75 , num2:  375
xnum1:  75
xnum2:  This will not print because it is not accessible
count:  15 , counter:  10
\end{lstlisting}

\subsubsection{Type of primitive data}

\begin{lstlisting}
"use strict";

let b  = false;
let array = [1,2,3,'hello', 3.02,  b];

var typeOfData = (array) =>{
  array.forEach((element) =>{
    console.log(element, 'is a ', typeof(element));
  })
}

typeOfData(array);

/*Output*/

1 'is a ' 'number'
2 'is a ' 'number'
3 'is a ' 'number'
hello is a  string
3.02 'is a ' 'number'
false 'is a ' 'boolean'

\end{lstlisting}

\subsubsection{Undefined and Null}

\begin{lstlisting}
"use strict";

let anUndefinedVariable;  //not initialized
let empty = null; //is empty (nothing)

console.log(anUndefinedVariable, empty);
console.log(typeof(anUndefinedVariable), typeof(empty));

/*Output*/

undefined null
undefined object

\end{lstlisting}

\subsubsection{Data containers}


\textbf{Array}

\begin{lstlisting}
"use strict";

/*
  We use arrays to contain multiple variables values instead of declaring a thousand of them.
*/

let array = ["John", "Doe", 34, "X", "USA", "Nevada", "Porsche 911", ["soccer", "volleyball", "chess"], ["python", "nim", "c", "java", "julia", "objective C", "SQL", "GraphQL", "JavaScript", "HTML5", "CSS3", "jQuery", "Machine Learning", "Bash"],"MIT", "In a relationship", ["Bali", "Singapore", "Hong Kong", "Thailand", "Mozambique", "Swaziland", "South Africa", "Lombark"], ["Electrical", "Computer"]];

array.forEach((element)=>{console.log(element)});

/*Output*/

John
Doe
34
X
USA
Nevada
Porsche 911
[ 'soccer', 'volleyball', 'chess' ]
[ 'python',
  'nim',
  'c',
  'java',
  'julia',
  'objective C',
  'SQL',
  'GraphQL',
  'JavaScript',
  'HTML5',
  'CSS3',
  'jQuery',
  'Machine Learning',
  'Bash' ]
MIT
In a relationship
[ 'Bali',
  'Singapore',
  'Hong Kong',
  'Thailand',
  'Mozambique',
  'Swaziland',
  'South Africa',
  'Lombark' ]
[ 'Electrical', 'Computer' ]

\end{lstlisting}

Add values into an empty array

\begin{lstlisting}
"use strict";

/*
  We use arrays to contain multiple variables values instead of declaring a thousand of them.
*/

let array = ["John", "Doe", 34, "X", "USA", "Nevada", "Porsche 911"];
let results = [] //empty array

/*
  add elements into array
  array.push(value)
*/

for(let i = 0; i < array.length; i++){
    results.push(array[i]);
}

console.log("Length of results[]: ", results.length);
console.log(results);

/*Output*/
Length of results[]:  7
[ 'John', 'Doe', 34, 'X', 'USA', 'Nevada', 'Porsche 911' ]

\end{lstlisting}


Removing elements from an array

\begin{lstlisting}
"use strict";

let array = ["John", "Doe", 34, "X", "USA", "Nevada", "Porsche 911"];
/*
 remove elements from an array
 array.pop(); //removes last value
*/
while(array.length > 0){
        array.pop();
}

console.log("Length of array: ", array.length);
console.log(array);

/*Output*/
Length of array:  0
[]

\end{lstlisting}

Removing the first elements by shifting the array.

\begin{lstlisting}

"use strict";

let array = ["John", "Doe", 34, "X", "USA", "Nevada", "Porsche 911"];
array.shift(); //shifts the array

console.log(array);

/*Output*/

[ 'Doe', 34, 'X', 'USA', 'Nevada', 'Porsche 911' ]

\end{lstlisting}

Deleting elements from an array.

\begin{lstlisting}
"use strict";

let array = ["cobol", "c#", ".NET", "Python"];

/*
  delete the first three elements of the array
*/

let languages_depricated = array.splice(0,3);
console.log(array, languages_depricated);

/*Output*/
[ 'Python' ] [ 'cobol', 'c#', '.NET' ]
\end{lstlisting}

Deleting elements from an array and mutating it.

\begin{lstlisting}
"use strict";

let array = ["cobol", "c#", ".NET", "Python"];

/*
  delete the first three elements of the array and mutating the array
  using splice()

	splice(0,3) - means:
	delete element from index 0 and delete 3 items
	if array = [1,2,3,4]
	splice(0,3)
			performs:
							[2,3,4]  - 1 : delete[1]
							[3,4] - 2 : delete[2]
							[4] - 3 : delete[3]
							all at index 0
			returns new array = [4]
*/

let deleted = array.splice(0,3, "Java", "C", "Nim", "Objective C", "Swing");
console.log(array, deleted);

/*Output*/
[ 'Java', 'C', 'Nim', 'Objective C', 'Swing', 'Python' ] [ 'cobol', 'c#', '.NET' ]
\end{lstlisting}

Some of Array methods you can use.

\begin{description}[align=left]
\item [forEach()] This method can help you to loop over array's items.
\begin{lstlisting}
const arr = [1, 2, 3, 4, 5, 6];

  arr.forEach(item => {
    console.log(item); // output: 1 2 3 4 5 6
  });
\end{lstlisting}
\item [includes()] This method check if array includes the item passed in the method.
\begin{lstlisting}
const arr = [1, 2, 3, 4, 5, 6];

 arr.includes(2); // output: true
 arr.includes(7); // output: false
\end{lstlisting}
\item [filter()] This method create new array with only elements passed condition inside the provided function.
\begin{lstlisting}
const arr = [1, 2, 3, 4, 5, 6];

  // item(s) greater than 3
  const filtered = arr.filter(num => num > 3);
  console.log(filtered); // output: [4, 5, 6]

  console.log(arr); // output: [1, 2, 3, 4, 5, 6]
\end{lstlisting}
\item [map()] This method create new array by calling the provided function in every element.The reduce() method applies a function against an accumulator and each element in the array (from left to right) to reduce it to a single value - MDN
\begin{lstlisting}
const arr = [1, 2, 3, 4, 5, 6];

 // add one to every element
 const oneAdded = arr.map(num => num + 1);
 console.log(oneAdded); // output [2, 3, 4, 5, 6, 7]

 console.log(arr); // output: [1, 2, 3, 4, 5, 6]
\end{lstlisting}
\item [reduce()] This method check if at least one of array's item passed the condition. If passed, it return 'true' otherwise 'false'.
\begin{lstlisting}
const arr = [1, 2, 3, 4, 5, 6];

 const sum = arr.reduce((total, value) => total + value, 0);
 console.log(sum); // 21
\end{lstlisting}
\item [some()] This method check if at least one of array's item passed the condition. If passed, it return 'true' otherwise 'false'.
\begin{lstlisting}
const arr = [1, 2, 3, 4, 5, 6];

// at least one element is greater than 4?
const largeNum = arr.some(num => num > 4);
console.log(largeNum); // output: true

// at least one element is less than or equal to 0?
const smallNum = arr.some(num => num <= 0);
console.log(smallNum); // output: false
\end{lstlisting}
\item [every()] This method check if all array's item passed the condition. If passed, it return 'true' otherwise 'false'.
\begin{lstlisting}
const arr = [1, 2, 3, 4, 5, 6];

  // all elements are greater than 4
  const greaterFour = arr.every(num => num > 4);
  console.log(greaterFour); // output: false

  // all elements are less than 10
  const lessTen = arr.every(num => num < 10);
  console.log(lessTen); // output: true
\end{lstlisting}
\item [sort()] This method used to arrange/sort array's item either ascending or descending order.
\begin{lstlisting}
const arr = [1, 2, 3, 4, 5, 6];
const alpha = ['e', 'a', 'c', 'u', 'y'];

// sort in descending order
descOrder = arr.sort((a, b) => a > b ? -1 : 1);
console.log(descOrder); // output: [6, 5, 4, 3, 2, 1]

// sort in ascending order
ascOrder = alpha.sort((a, b) => a > b ? 1 : -1);
console.log(ascOrder); // output: ['a', 'c', 'e', 'u', 'y']
\end{lstlisting}
\item [Array.from()] This change all thing that are array-like or iterable into true array especially when working with DOM, so that you can use other array methods like reduce, map, filter and so on.
\newline\newline
code 1
\begin{lstlisting}
const name = 'frugence';
const nameArray = Array.from(name);

console.log(name); // output: frugence
console.log(nameArray); // output: ['f', 'r', 'u', 'g', 'e', 'n', 'c', 'e']
\end{lstlisting}
code 2
\begin{lstlisting}
// I assume that you have created unorder list of items in our html file.

const lis = document.querySelectorAll('li');
const lisArray = Array.from(document.querySelectorAll('li'));

// is true array?
console.log(Array.isArray(lis)); // output: false
console.log(Array.isArray(lisArray));  // output: true
\end{lstlisting}
\item [Array.of()] This create array from every arguments passed into it.
\begin{lstlisting}
const nums = Array.of(1, 2, 3, 4, 5, 6);
console.log(nums); // output: [1, 2, 3, 4, 5, 6]
\end{lstlisting}
\end{description}

\newpage
\textbf{Dictionary}

\begin{lstlisting}

"use strict";

let data  = {

    "first name": "John",
    "last name" : "Doe",
    "age"       : 34,
    "company"   : "X",
    "country"   : "USA",
    "State"     : "Nevada",
    "car"       : "Porsche 911",
    "hobby"     : ["soccer", "volleyball", "chess"],
    "polyglot"  : ["python", "nim", "c", "java", "julia", "objective C", "SQL", "GraphQL", "JavaScript", "HTML5", "CSS3", "jQuery", "Machine Learning", "Bash"],
    "university" : "MIT",
    "status"     : "in a relationship",
    "travels"    : ["Bali", "Singapore", "Hong Kong", "Thailand", "Mozambique", "Swaziland", "South Africa", "Lombark"],
    "Degrees"   : ["Electrical", "Computer"]


}

console.log(data);


/*Output*/

{ 'first name': 'John',
  'last name': 'Doe',
  age: 34,
  company: 'X',
  country: 'USA',
  State: 'Nevada',
  car: 'Porsche 911',
  hobby: [ 'soccer', 'volleyball', 'chess' ],
  polyglot:
   [ 'python',
     'nim',
     'c',
     'java',
     'julia',
     'objective C',
     'SQL',
     'GraphQL',
     'JavaScript',
     'HTML5',
     'CSS3',
     'jQuery',
     'Machine Learning',
     'Bash' ],
  university: 'MIT',
  status: 'in a relationship',
  travels:
   [ 'Bali',
     'Singapore',
     'Hong Kong',
     'Thailand',
     'Mozambique',
     'Swaziland',
     'South Africa',
     'Lombark' ],
  Degrees: [ 'Electrical', 'Computer' ] }

\end{lstlisting}


\subsubsection{Blackjack project (PluralSight)}

\begin{lstlisting}
/*
  Blackjack game of cards
*/

let card1 = "Ace of Spades", card2 = "Ten of hearts";
let cards = [card1, card2];

console.log("Welcome to Blackjack");
console.log("You are dealt: ");
cards.forEach((element) => {
    console.log("\t" + element);
})

/*Output*/
Welcome to Blackjack
You are dealt:
        Ace of Spades
        Ten of hearts
\end{lstlisting}

For loops, Arrays

\begin{lstlisting}
/*
  Blackjack game of cards
*/

let suits = ["Heart", "Clubs", "Diamonds", "Spades"];
let values = ["Ace", "King", "Queen", "Jack", "Ten", "Nine", "Eight", "Seven", "Six", "Five", "Four", "Three", "Two"];


let deck = []

for(let suitIdx = 0; suitIdx < suits.length; suitIdx++){
    for(let valueIdx = 0; valueIdx < values.length; valueIdx++){
        deck.push(values[valueIdx] + ' of ' + suits[suitIdx]);
    }
}

console.log(deck);

/*Output*/
....
 'Four of Spades',
 'Three of Spades',
 'Two of Spades' ]
\end{lstlisting}

%----------------------------------------------------------------------------------------
%	BIBLIOGRAPHY
%----------------------------------------------------------------------------------------

\printbibliography[title={Bibliography}] % Print the bibliography, section title in curly brackets

%----------------------------------------------------------------------------------------

\end{document}

\documentclass[10pt, a4paper, twocolumn]{article}

\input{structure.tex}
\title{JavaScript}

\author{
	\authorstyle{Boitumelo Phetla\textsuperscript{1,2,3} (PluralSight Google ScholarShip)\textsuperscript{2,3}} % Authors
	\newline\newline % Space before institutions
	\textsuperscript{1}\institution{Universidad Nacional Autónoma de México, Mexico City, Mexico}\\ % Institution 1
	\textsuperscript{2}\institution{University of Texas at Austin, Texas, United States of America}\\ % Institution 2
	\textsuperscript{3}\institution{\texttt{LaTeXTemplates.com}} % Institution 3
}


\date{\today} % Add a date here if you would like one to appear underneath the title block, use \today for the current date, leave empty for no date


\begin{document}
\maketitle % Print the title
\thispagestyle{firstpage} % Apply the page style for the first page (no headers and footers)

%----------------------------------------------------------------------------------------
%	ABSTRACT
%----------------------------------------------------------------------------------------

\lettrineabstract{JavaScript,  is a lightweight interpreted or just-in-time compiled programming language with first-class functions. While it is most well-known as the scripting language for Web pages, many non-browser environments also use it, such as Node.js, Apache CouchDB and Adobe Acrobat.}

%----------------------------------------------------------------------------------------
%	ARTICLE CONTENTS
%----------------------------------------------------------------------------------------

\section{Simplistic JavaScript 1}

\subsection{Command-line based programming}

A simple project: 

\begin{lstlisting}
$bash: touch {index.html,script.js,style.css}
$bash: tree
	__________	index.html
	__________	script.js
	__________	style.css
\end{lstlisting}

Include the script (\textbf{javascript}) and the page styling script (\textbf{cascading stylesheet}) files into the \textit{index.html}.

\begin{lstlisting}
	<head>
			<script src="path/*.js"></script>
			<link rel="stylesheet" href="path/*.css">
	</head>
\end{lstlisting}

Add some simple HTML markup code and launch a live-server of the code. 

\begin{lstlisting}
	<!DOCTYPE>
	<html>
			<head>
					<script src="script.js"></script>
					<link rel="stylesheet" href="style.css">
			</head>
					<body>
							<div id="header">
									<h1>Welcome to JavaScript</h1>
							</div>
					</body>
	</html>
\end{lstlisting}

Launch the command-line (Terminal)

\begin{lstlisting}
$bash: live-server
\end{lstlisting}

\begin{figure}[h!]
	\includegraphics[width=\linewidth]{liveserver.png} % Figure image
	\caption{Live-server} % Figure caption
	\label{ls} % Label for referencing with \ref{bear}
\end{figure}

\subsection{\href{http://plnkr.co}{Plunker}}

Or create an account on  \href{http://plnkr.co/edit/?p=catalogue}{Plunker}.  Plunker sets up your working environment for you.

\begin{figure}[h!]
	\includegraphics[width=0.89\linewidth]{plunker.png} % Figure image
	\caption{Plunker} % Figure caption
	\label{plnkr} % Label for referencing with \ref{bear}
\end{figure}

\subsection{\href{https://electronjs.org}{Electron}}

Watch this video \href{https://www.youtube.com/watch?v=8YP_nOCO-4Q&feature=youtu.be}{Electron}.

\begin{lstlisting}
# Clone the Quick Start repository
$ git clone https://github.com/electron/electron-quick-start

# Go into the repository
$ cd electron-quick-start

# Install the dependencies and run
$ npm install && npm start
\end{lstlisting}

\begin{figure}[h!]
	\includegraphics[width=\linewidth]{electron.png} % Figure image
	\caption{Electron} % Figure caption
	\label{elec} % Label for referencing with \ref{bear}
\end{figure}

\begin{lstlisting}
$bash: mkdir Electron1; cd Electron1; npm init
  1 {
  2   "name": "electron1",
  3   "version": "1.0.0",
  4   "description": "First App",
  5   "main": "index.js",
  6   "scripts": {
  7     "test": "echo \"Error: no test specified\" && exit 1"
  8   },
  9   "keywords": [
 10     "Electron"
 11   ],
 12   "author": "Boitumelo Phetla",
 13   "license": "ISC"
 14 }
\end{lstlisting}

At this point, you'll need to install electron itself. The recommended way of doing so is to install it as a development dependency in your app, which allows you to work on multiple apps with different Electron versions. To do so, run the following command from your app's directory:

\begin{lstlisting}
$bash: npm install --save-dev electron
$bash: tree -L 1
			.
			|____________node_modules
			|____________package-lock.json
			|____________package.json

1 directory, 2 files
\end{lstlisting}

All APIs and features found in Electron are accessible through the electron module, which can be required like any other Node.js module:

\begin{lstlisting}
const electron = require('electron')
\end{lstlisting}

To avoid any huddles, try this simple example.

\begin{lstlisting}
# Clone the repository
$ git clone https://github.com/electron/electron-quick-start
# Go into the repository
$ cd electron-quick-start
# Install dependencies
$ npm install
# Run the app
$ npm start
\end{lstlisting}

\subsection{\href{https://www.meteor.com/install}{Meteor}}


\begin{figure}[h!]
	\includegraphics[width=\linewidth]{meteor.png} % Figure image
	\caption{Meteor} % Figure caption
	\label{mtr} % Label for referencing with \ref{bear}
\end{figure}

To create the app, open your terminal and type:

\begin{lstlisting}
$bash: meteor create simple-todos

output:

Created a new Meteor app in 'simple-todos'.                                        

To run your new app:                          
  cd simple-todos                             
  meteor                                      
                                              
If you are new to Meteor, try some of the learning resources here:
  https://www.meteor.com/tutorials            
                                              
To start with a different app template, try one of the following:

  meteor create --bare    # to create an empty app
  meteor create --minimal # to create an app with as few Meteor packages as possible
  meteor create --full    # to create a more complete scaffolded app

\end{lstlisting}


%------------------------------------------------

\subsection{JavaScript Editors (IDEs)}



\begin{itemize}
	\item \href{http://plnkr.co/edit/?p=catalogue}{Plunker}
	\item Second item in a list
	\item Third item in a list
\end{itemize}



%----------------------------------------------------------------------------------------
%	BIBLIOGRAPHY
%----------------------------------------------------------------------------------------

\printbibliography[title={Bibliography}] % Print the bibliography, section title in curly brackets

%----------------------------------------------------------------------------------------

\end{document}
